%
% This is a borrowed LaTeX template file for lecture notes for CS267,
% Applications of Parallel Computing, UCBerkeley EECS Department.
% Now being used for CMU's 10725 Fall 2012 Optimization course
% taught by Geoff Gordon and Ryan Tibshirani.  When preparing 
% LaTeX notes for this class, please use this template.
%
% To familiarize yourself with this template, the body contains
% some examples of its use.  Look them over.  Then you can
% run LaTeX on this file.  After you have LaTeXed this file then
% you can look over the result either by printing it out with
% dvips or using xdvi. "pdflatex template.tex" should also work.
%

\documentclass[twoside]{article}


\setlength{\oddsidemargin}{0.25 in}
\setlength{\evensidemargin}{-0.25 in}
\setlength{\topmargin}{-0.6 in}
\setlength{\textwidth}{6.5 in}
\setlength{\textheight}{8.5 in}
\setlength{\headsep}{0.75 in}
\setlength{\parindent}{0 in}
\setlength{\parskip}{0.1 in}

%
% ADD PACKAGES here:
%

\usepackage{amsmath,amsfonts,graphicx}
\usepackage{booktabs}
\usepackage{multirow}
\usepackage{xcolor}
\usepackage{float}

%
% The following commands set up the lecnum (lecture number)
% counter and make various numbering schemes work relative
% to the lecture number.
%
\newcounter{lecnum}
\renewcommand{\thepage}{\thelecnum-\arabic{page}}
\renewcommand{\thesection}{\thelecnum.\arabic{section}}
\renewcommand{\theequation}{\thelecnum.\arabic{equation}}
\renewcommand{\thefigure}{\thelecnum.\arabic{figure}}
\renewcommand{\thetable}{\thelecnum.\arabic{table}}

%
% The following macro is used to generate the header.
%
\newcommand{\lecture}[4]{
   \pagestyle{myheadings}
   \thispagestyle{plain}
   \newpage
   \setcounter{lecnum}{#1}
   \setcounter{page}{1}
   \noindent
   \begin{center}
   \framebox{
      \vbox{\vspace{2mm}
    \hbox to 6.28in { {\bf UCSB CS 291D: Blockchains and Cryptocurrencies
	\hfill Fall 2020} }
       \vspace{4mm}
       \hbox to 6.28in { {\Large \hfill Lecture #1: #2  \hfill} }
       \vspace{2mm}
       \hbox to 6.28in { {\it Lecturer: #3 \hfill Scribes: #4} }
      \vspace{2mm}}
   }
   \end{center}
   \markboth{Lecture #1: #2}{Lecture #1: #2}

%   {\bf Note}: {\it LaTeX template courtesy of UC Berkeley EECS dept.}

%   {\bf Disclaimer}: {\it These notes have not been subjected to the
%   usual scrutiny reserved for formal publications.  They may be distributed
%   outside this class only with the permission of the Instructor.}
%   \vspace*{4mm}
}
%
% Convention for citations is authors' initials followed by the year.
% For example, to cite a paper by Leighton and Maggs you would type
% \cite{LM89}, and to cite a paper by Strassen you would type \cite{S69}.
% (To avoid bibliography problems, for now we redefine the \cite command.)
% Also commands that create a suitable format for the reference list.
\renewcommand{\cite}[1]{[#1]}
\def\beginrefs{\begin{list}%
        {[\arabic{equation}]}{\usecounter{equation}
         \setlength{\leftmargin}{2.0truecm}\setlength{\labelsep}{0.4truecm}%
         \setlength{\labelwidth}{1.6truecm}}}
\def\endrefs{\end{list}}
\def\bibentry#1{\item[\hbox{[#1]}]}

%Use this command for a figure; it puts a figure in wherever you want it.
%usage: \fig{NUMBER}{SPACE-IN-INCHES}{CAPTION}
\newcommand{\fig}[3]{
			\vspace{#2}
			\begin{center}
			Figure \thelecnum.#1:~#3
			\end{center}
	}
% Use these for theorems, lemmas, proofs, etc.
\newtheorem{theorem}{Theorem}[lecnum]
\newtheorem{lemma}[theorem]{Lemma}
\newtheorem{proposition}[theorem]{Proposition}
\newtheorem{claim}[theorem]{Claim}
\newtheorem{corollary}[theorem]{Corollary}
\newtheorem{definition}[theorem]{Definition}
\newenvironment{proof}{{\bf Proof:}}{\hfill\rule{2mm}{2mm}}

% **** IF YOU WANT TO DEFINE ADDITIONAL MACROS FOR YOURSELF, PUT THEM HERE:

\newcommand\E{\mathbb{E}}

\begin{document}
%FILL IN THE RIGHT INFO.
%\lecture{**LECTURE-TITLE**}{**DATE**}{**LECTURER**}{**SCRIBE**}
\lecture{5}{Consensus III: Speed Up Consensus using Randomness}{Derek Leung}{Dipanjan Das, Chinmay Sonar}
%\footnotetext{These notes are partially based on those of Nigel Mansell.}

% **** YOUR NOTES GO HERE:

% Some general latex examples and examples making use of the
% macros follow.  
%**** IN GENERAL, BE BRIEF. LONG SCRIBE NOTES, NO MATTER HOW WELL WRITTEN,
%**** ARE NEVER READ BY ANYBODY.

\section{Desirable properties of a cryptocurrency system} % Don't be this informal in your notes!
For a cryptocurrency system to be practical and usable, it should meet the following requirements:

\begin{itemize}
	\item \textbf{Secure:}
	\begin{itemize}
		\item \textbf{Theft:} Only owner should be able to spend the money, and no one else.
		\item \textbf{Consensus:} If money is spent, everybody should agree.
		In other words, all the parties must agree to the order of transactions that have taken place in the system.
		Also, no money can be double spent.
	\end{itemize}
	\item \textbf{Distributed:} The transaction log is distributed among all the parties using the system.
	\item \textbf{Anonymity:} Only transactions are public, but not the identities of the sender, or the receiver.
	\item \textbf{Scalable:} Achieves high throughput in terms of the number of transactions executed per unit time.
	\item \textbf{Efficient:} The storage space required to store the transaction log, and the bandwidth and the processing power required to execute the distributed protocol should be optimal.
	\item \textbf{Fault tolerance:} Should be resilient to network interruptions, or faulty, or adversarial nodes.
	\item \textbf{Fairness:} Like in any trade, the payer receives~\cite{1} the item/service she has paid for.
\end{itemize}


\section{The Byzantine Generals Problem (BGP) -- Revisited}
The Byzantine Generals Problem allegorically models a fundamental problems in the distributed systems where $n$ nodes need to come to a consensus under the following assumptions:

\begin{itemize}
	\item \textbf{Authenticated channel:} Receivers know the source of the message they receive.
	\item \textbf{Bad actors:} There are $f$ bad actors present in the system who try to either break the consistency of the system, or its availability.
	\item \textbf{Delay:} Message delays are unbounded, \textit{i.e.}, the network may be unreliable.
\end{itemize}

\begin{theorem}
	It is impossible to solve the problem unless $3f + 1 \leq n$.
	In other words, in a synchronous network with authenticated channel where $f$ actors are corrupt, no solution will work unless there are more than 3f actors present in the network. 
\end{theorem}

\subsection{Proving impossibility by example}
Let's assume $n=3$ actors with $f=1$ corrupt actor want to agree on a bit $b \in \{0,1\}$.
Table~\ref{tab:bgp_impossbility_proof1} and Table~\ref{tab:bgp_impossbility_proof2} show the bits sent and received by different actors.
Actor \textbf{B} is dishonest.
In the first case, it sends bit $0$ to \textbf{A}, and bit $1$ to \textbf{C}.
Therefore, \textbf{A} and \textbf{C} agree on different bits---$0$ and $1$ respectively.
In the seconds case, it sends no message at all---thus preventing \textbf{A} and \textbf{C} reach any consensus.
Either one breaks the consistency of the system.

\begin{minipage}{.45\textwidth} %
\begin{table}[H]
	\centering
	\renewcommand{\arraystretch}{1.5}
	\begin{tabular}{cc|ccc}
		&  & \multicolumn{3}{c}{\textbf{Sent}} \\ \cline{3-5} 
		& \textbf{} & \textbf{A} & \multicolumn{1}{c}{\textbf{B}} & \multicolumn{1}{c}{\textbf{C}} \\ \midrule
		\multicolumn{1}{l|}{\multirow{3}{*}{\rotatebox{90}{\textbf{Received}}}} & \textbf{A} & 1 & \textcolor{red}{0} & 0 \\
		\multicolumn{1}{l|}{} & \textbf{B} & 1 & --- & 0 \\
		\multicolumn{1}{l|}{} & \textbf{C} & 1 & \textcolor{red}{1} & 0 \\ 
	\end{tabular}
	\caption{Impossibility of the Byzantine Generals Problem with less than $3f + 1$ actors.
		\textbf{B}, who sends two different messages to \textbf{A} and \textbf{C}, is the bad actor here.}
	\label{tab:bgp_impossbility_proof1}
\end{table}
\end{minipage} %
\hfill
\begin{minipage}{.45\textwidth} %
\begin{table}[H]
	\centering
	\renewcommand{\arraystretch}{1.5}
	\begin{tabular}{cc|ccc}
		&  & \multicolumn{3}{c}{\textbf{Sent}} \\ \cline{3-5} 
		& \textbf{} & \textbf{A} & \multicolumn{1}{c}{\textbf{B}} & \multicolumn{1}{c}{\textbf{C}} \\ \midrule
		\multicolumn{1}{l|}{\multirow{3}{*}{\rotatebox{90}{\textbf{Received}}}} & \textbf{A} & 1 & \textcolor{red}{---} & 0 \\
		\multicolumn{1}{l|}{} & \textbf{B} & 1 & --- & 0 \\
		\multicolumn{1}{l|}{} & \textbf{C} & 1 & \textcolor{red}{---} & 0 \\ 
	\end{tabular}
	\caption{Impossibility of the Byzantine Generals Problem with less than $3f + 1$ actors.
		\textbf{B}, who sends no message to \textbf{A} and \textbf{C}, is the bad actor here.}
	\label{tab:bgp_impossbility_proof2}
\end{table}
\end{minipage}


\subsection{Practical Byzantine Fault Tolerance (PBFT)}
Practical Byzantine Fault Tolerance (PBFT) is a leader based, two-round protocol for the Byzantine Generals Problem.
Let's assume there are $n$ actors of which $f$ are corrupt, where $3f + 1 \leq n$.
The steps of the algorithm are:

\begin{enumerate}
	\item Some leader proposes a transaction $T$ by broadcasting it to others
	\item When machine $x$ hears the transaction $T$, it acknowledges the receipt by broadcasting $\texttt{ACK}(x, T)$
	\item When machine $y$ hears quorum of $\texttt{ACK}(x, T)$, it acknowledges the receipt by broadcasting $\texttt{ACK}(y, \text{quorum of } \texttt{ACK}(x, T))$.
	A \textit{quorum} is defined as $2f + 1$ acknowledgment messages $\texttt{ACK}(x, T)$ for the same transaction $T$
	\item When machine $z$ hears $\texttt{ACK}(y, \text{quorum of } \texttt{ACK}(x, T))$, it knows $T$ is committed
\end{enumerate}


\subsection{PBFT by example}
Let's assume there are $n=4$ actors out of which $f=1$ are corrupt.
Therefore, quorum ($2f + 1$) is $3$ messages/actor.
There actors are $\{x-1, x_2, x_3, x_4\}$.
We assume the leader is \textit{not} dishonest.
Let $x_2$ be the bad actor here.

\begin{enumerate}
	\item \textbf{Round 0 (Proposal): } $x_2$ proposes a transaction $T$.
	$x_i$, where $i \in \{1, 3, 4\}$, listens to this broadcast
	\item \textbf{Round 1: } $x_i$ acknowledges the receipt by broadcasting $\texttt{ACK}(x_i, T)$, where $i \in \{1, 3, 4\}$
	\item \textbf{Round 2: } $x_i$ re-acknowledges the acknowledgment messages from the round $1$ by broadcasting $\texttt{ACK}(x_i, \texttt{ACK}(\{x_j\}, T))$, where $i, j \in \{1, 3, 4\}$
	\item \textbf{Finalization:} $x_i$, where $i \in \{1, 3, 4\}$, commits to $T$
\end{enumerate}


\subsection{Decentralized BFT}
PBFT can't be applied naively to a decentralized setting like cryptocurrency system because of the following issues:

\begin{itemize}
	\item The number of actors $n$ is not known before the protocol is executed.
	Moreover, $n$ changes dynamically as nodes are free to join or leave the network at any time.
	\item The complexity of PBFT is $O(n^2)$.
	In a practical system, $n$ could be huge.
	For example, if there are $1$ million nodes present in the network, approximately $1$ trillion messages are exchanged.
\end{itemize}

The idea is to fairly sample $n^\prime \subset n$ nodes to bring the final complexity down to $O(nn^\prime)$.
Speaking of \textit{fairness}, different protocols adapt different sampling strategies.
In Bitcoin, Nakamoto consensus protocol proposes a technique called \textit{proof-of-work} that regulates a machine's power in the system as a function of its computational power.
In other words, the more processing power a machine has, it has got more power in the system.
However, \textit{proof-of-work} is resource demanding, hence not very efficient.
\textit{Proof-of-stake} is a viable alternative that regulates a machine's power in the system as a function of the money being held in the account.
In general, finding a correct definition of fairness is one of the major challenges in developing a new system.

In this lecture, we focus on \textit{proof of stake}. 
We study a secure protocol which achieves efficiency by sampling.

\section{Sortition}

Properties of the Sortition procedure:
\begin{enumerate}
    \item Non-interactive
    \item Stake should be representative of the power
    \item Efficient
    \item Forward secure (ephemeral signatures)
    \item Deniel of service resistance
\end{enumerate}

Next, we present a particular protocol which uses sortition, and has first three properties.

\subsection{Strawman}
\label{subsec:strawman-naive}

Each actor/participant in the system follows the protocol described below:
\begin{enumerate}
    \item Sample $s_{i} \leftarrow{}$ random bit-string of length $n$.
    \item Broadcast $s_{i}$. Store the random bit strings $s_{i}'$ for other actors in the system by listening on the network.
    \item Compute $S = s_{1} \bigoplus s_{2} \bigoplus \cdots \bigoplus s_{n}$
    \item If and only if $H(S, i) * \frac{\mbox{money owned by i}}{\mbox{total money in the system}} > c$; actor i is selected for this round.
\end{enumerate}

In step $4$, $H(S, i)$ is a hash of any function of integers $S$ and $i$ e.g., one can compute $H(S \bigoplus i)$, and $c$ is a constant threshold (decided at the start of the prototcol).

Observe that the above protocol is \emph{interactive} (as an actor sends it's $s_{i}$ on the network), protocol is \emph{efficient} as everyone need to send just one message.
And, the protocol is \emph{stake representive} as the money owned by an actor appears in the numerator of computation in step $4$ which essentially detemines
if the agent participates in a particular round.

\subsection{Issues with the naive implementation of Strawman}

\begin{itemize}
    \item If we set the threshold $c$ to be very high, then some rounds might not have any participant, hence, the \emph{availability} is compromised.
    \item Security concers: 
    \begin{itemize}
        \item An actor can wait for other all other actors to send their $s_{i}'s$, and then choose its own $s_{i}$ adversarially to control the value of $S$. Note that this issue occurs as we assume an asychronous system.
    \end{itemize}
\end{itemize}

\subsection{Modifications over Strawman}

We make use of \emph{Verifiable Random Function} (VRF) introduced by Micali, Rabin and Vadhan~\cite{2}.
Their construction uses an involved analysis using a Schnorr proof to show security. This proof is beyond the scope of this lecture.

VRF consists of 3 algorithms:
\begin{enumerate}
	\item Keygen(security parameter) $\rightarrow{} ~~ (pk,sk)$
	\item Hash(m, sk) $\rightarrow$ $h,sig\_(m,pk)$
	\item verify$(m, pk, sig)$ $\rightarrow$ retuns if the signature was produced by sk matching $pk$ on $m$
\end{enumerate}

Note that in step 2, $h$ is an additional part compared to the normal public key based signature schemes.
We also note that $Hash = SHA256(Sign(m,sk)) \rightarrow{} sign(m,sk)$ here we have to use a  cryptogrphic hash function like SHA256, since sig has to be a deteministic algorithm.

Hence, with VRF, we modify our previous protocol by changing its last step as follows:
\begin{enumerate}
	\item Sample $s_{i} \leftarrow{}$ random bitstring of length $n$.
    \item Broadcast $s_{i}$. Store the random bit strings $s_{i}'$ for other actors in the system by listening on the network.
    \item Compute $S = s_{1} \bigoplus s_{2} \bigoplus \cdots \bigoplus s_{n}$
    \item Consists of three parts:
    \begin{itemize}
		\item VRF\_Hash(i $\bigoplus$ ACK(sk, T), sk) $\rightarrow{}$ h, sig
		\item Check if $h \times \frac{\mbox{money owned by i}}{\mbox{total money in the system}} > c$
		\item If the above condition evaluates to true, transmit $(ACK(sk, T)), pk, sig$
	\end{itemize}
	Each actor $i$ follows the above procedure.
\end{enumerate}

If some actor transmits $(ACK(sk, T)), pk, sig$, then other actors in the system needs to verify the message.

\textbf{Verify procedure}:
	\begin{itemize}
		\item Check if $h \times \frac{\mbox{money owned by i}}{\mbox{total money in the system}} > c$
		\item Check VRF\_verify(ACK(sk,T), pk, sig)
	\end{itemize}

\textbf{Remarks}:
	\begin{itemize}
		\item Quorum size is roughly around $1000$.
		\item We want each agent to run VRF\_hash exactly once, i.e. no one should have an incentive to run it more than once to obtain a desired hash value $h$.
		Otherwise, this system will become similar to the Proof of Work (computation intensive).
		\item To achieve this, the VRF\_Hash function given message (ACK in the above example) and sk produces the same hash value. 
		Hence, adversary do not have any incentive to run VFR\_Hash multiple times.
		\item In some sense, both POW abd POS favors rich agents. This reiterates the issue of choosing a ``fair'' protocol.
		Some of the other notion considered in the literature are:
		\begin{itemize}
			\item Proof of ``personhood''
			\item Proof of storage
			\item Proof of delay
		\end{itemize}
		\item Setting a threshold is tricky as we are faced with the tradeoff: If we set threshold too low, we might not abide to 
		our guarantee of share proportional to the stake. Otherwise, if the threshold is set too high, then quorum size has to be reduced.
		\item The above protocol do not have \emph{Anonymity} since the stake table is public.
	\end{itemize}
	
\subsection{Practical considerations}
While implementing the protocol as above, one might need to consider following scenarios (this is an informal discussion, and the list is by no means complete):
\begin{itemize}
	\item Enough people should sign up for a new system.
	\item Account compromise (hacking) -- Important to keep secret key safe for POS
	\item ``Fairness'' -- initially, with less people in the system, someone might get too rich
	\item Faulty implementation: implementation bugs can subvert any guarantee such as security or availability.
\end{itemize}

\section*{References}
\beginrefs
\bibentry{1}{\sc Jian Liu} and {\sc Wenting Li}, and {\sc Ghassan O. Karame}, and {\sc N. Asokan}
``Toward Fairness of Cryptocurrency Payments,''
{\it IEEE S\&P},
2018.

\bibentry{2}{\sc Silvio Micali} and {\sc Michael Rabin}, and {\sc Salil Vadhan}.
``Verifiable Random Functions'', 
40th annual symposium on foundations of computer science (cat. No. 99CB37039),
{\it IEEE S\&P},
1999.

\endrefs

% **** THIS ENDS THE EXAMPLES. DON'T DELETE THE FOLLOWING LINE:

\end{document}

