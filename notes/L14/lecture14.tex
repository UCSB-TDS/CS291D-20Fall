%
% This is a borrowed LaTeX template file for lecture notes for CS267,
% Applications of Parallel Computing, UCBerkeley EECS Department.
% Now being used for CMU's 10725 Fall 2012 Optimization course
% taught by Geoff Gordon and Ryan Tibshirani.  When preparing 
% LaTeX notes for this class, please use this template.
%
% To familiarize yourself with this template, the body contains
% some examples of its use.  Look them over.  Then you can
% run LaTeX on this file.  After you have LaTeXed this file then
% you can look over the result either by printing it out with
% dvips or using xdvi. "pdflatex template.tex" should also work.
%

\documentclass[twoside]{article}
\setlength{\oddsidemargin}{0.25 in}
\setlength{\evensidemargin}{-0.25 in}
\setlength{\topmargin}{-0.6 in}
\setlength{\textwidth}{6.5 in}
\setlength{\textheight}{8.5 in}
\setlength{\headsep}{0.75 in}
\setlength{\parindent}{0 in}
\setlength{\parskip}{0.1 in}

%
% ADD PACKAGES here:
%

\usepackage{amsmath,amsfonts,graphicx}

%
% The following commands set up the lecnum (lecture number)
% counter and make various numbering schemes work relative
% to the lecture number.
%
\newcounter{lecnum}
\renewcommand{\thepage}{\thelecnum-\arabic{page}}
\renewcommand{\thesection}{\thelecnum.\arabic{section}}
\renewcommand{\theequation}{\thelecnum.\arabic{equation}}
\renewcommand{\thefigure}{\thelecnum.\arabic{figure}}
\renewcommand{\thetable}{\thelecnum.\arabic{table}}

%
% The following macro is used to generate the header.
%
\newcommand{\lecture}[4]{
   \pagestyle{myheadings}
   \thispagestyle{plain}
   \newpage
   \setcounter{lecnum}{#1}
   \setcounter{page}{1}
   \noindent
   \begin{center}
   \framebox{
      \vbox{\vspace{2mm}
    \hbox to 6.28in { {\bf UCSB CS 291D: Blockchains and Cryptocurrencies
	\hfill Fall 2020} }
       \vspace{4mm}
       \hbox to 6.28in { {\Large \hfill Lecture #1: #2  \hfill} }
       \vspace{2mm}
       \hbox to 6.28in { {\it Lecturer: #3 \hfill Scribes: #4} }
      \vspace{2mm}}
   }
   \end{center}
   \markboth{Lecture #1: #2}{Lecture #1: #2}

%   {\bf Note}: {\it LaTeX template courtesy of UC Berkeley EECS dept.}

%   {\bf Disclaimer}: {\it These notes have not been subjected to the
%   usual scrutiny reserved for formal publications.  They may be distributed
%   outside this class only with the permission of the Instructor.}
%   \vspace*{4mm}
}
%
% Convention for citations is authors' initials followed by the year.
% For example, to cite a paper by Leighton and Maggs you would type
% \cite{LM89}, and to cite a paper by Strassen you would type \cite{S69}.
% (To avoid bibliography problems, for now we redefine the \cite command.)
% Also commands that create a suitable format for the reference list.
\renewcommand{\cite}[1]{[#1]}
\def\beginrefs{\begin{list}%
        {[\arabic{equation}]}{\usecounter{equation}
         \setlength{\leftmargin}{2.0truecm}\setlength{\labelsep}{0.4truecm}%
         \setlength{\labelwidth}{1.6truecm}}}
\def\endrefs{\end{list}}
\def\bibentry#1{\item[\hbox{[#1]}]}

%Use this command for a figure; it puts a figure in wherever you want it.
%usage: \fig{NUMBER}{SPACE-IN-INCHES}{CAPTION}
\newcommand{\fig}[3]{
			\vspace{#2}
			\begin{center}
			Figure \thelecnum.#1:~#3
			\end{center}
	}
% Use these for theorems, lemmas, proofs, etc.
\newtheorem{theorem}{Theorem}[lecnum]
\newtheorem{lemma}[theorem]{Lemma}
\newtheorem{proposition}[theorem]{Proposition}
\newtheorem{claim}[theorem]{Claim}
\newtheorem{corollary}[theorem]{Corollary}
\newtheorem{definition}[theorem]{Definition}
\newenvironment{proof}{{\bf Proof:}}{\hfill\rule{2mm}{2mm}}

% **** IF YOU WANT TO DEFINE ADDITIONAL MACROS FOR YOURSELF, PUT THEM HERE:

\newcommand\E{\mathbb{E}}

\begin{document}
%FILL IN THE RIGHT INFO.
%\lecture{**LECTURE-TITLE**}{**DATE**}{**LECTURER**}{**SCRIBE**}
\lecture{14}{ZCash}{Shumo Chu}{Aarti Jivrajani}
%\footnotetext{These notes are partially based on those of Nigel Mansell.}


In this class, we are going to take a look at ZCash which is an application of zk-SNARKs. 

Bitcoin tries to give a solution to the problem of decentralized payment. But Bitcoin has a privacy issue - unlike conventional payment systems where the transactions are private. Bitcoin is called \textit{pseudo-anonymous} but that is not synonymous to privacy. It is possible to link the actual identity of the user to the public key using transactional graph analysis and machine learning techniques. 

\section{Predecessors of ZCash}
\subsection{Mixer}
\begin{itemize}
    \item It is a kind of centralized party. Each mixer has a \textit{pool} into which money can be deposited and from which money can be withdrawn. 
    \item Money can be deposited and withdrawn multiple times. And this amount need not be the same. 
    \item Let the deposit amount be $v_1, v_2 \text{ and } v_3$ and withdrawal amount be $v_1', v_2' \text{ and } v_3'$. Mixer expects the following property to satisfy:
    $$v_1 + v_2 + v_3 = v_1' + v_2' + v_3'$$
    \item For each $v_i \text{ and } v_i'$, different addresses can be used. Let these addresses be \\ $PK_1, PK_2, PK_3.... \text{ and } PK_1', PK_2', PK_3'$
    \item From an adversary point of view, the mixer pool acts like a shield and the addresses (especially during periods of high volume) are not visible. At the least, tracking the transactions becomes very difficult in this case.
\end{itemize}

\textbf{Drawbacks}
\begin{itemize}
    \item A centralized party - in this case the pool - needs to be trusted. This voids the point of using a cryptocurrency altogether. 
    \item The pool also acts like a centralized target for attacks. 
    \item A \textit{time latency} needs to be introduced between transaction \textit{in} and transaction \textit{out}. This is to prevent the transactions from being decoded using the time information.
\end{itemize}
Consider a bitcoin called \textit{basecoin}. In case of mixer, we get the basecoin \textit{in} and basecoin \textit{out}. In order to obfuscate the transaction graph, we need some kind of \textit{processing bench}. \\

Herein lies the motivation for ZCash - instead of a centralized trusted party, can we implement a \textit{mixer} kind of pool which embeds consensus within it. Can we also convince every participant in the consensus round, that a particular transaction is a valid transaction and at the same time, there should be no information leakage. 

\section{ZCash}
After taking a closer look at the system described above, using zk-SNARK seems to be the right choice to implement such a payment system. 
\subsection{Design Goal}
\begin{itemize}
    \item Decentralized Anonymous Payment Scheme.
    \item At any given time, there is a unique snapshot of the ledger state and is available to everyone in the consensus. Here, a \textit{ledger} is a sequence of append-only transactions. 
    \item ZCash supports 2 types of transactions - \underline{base transactions} and \underline{ZCash transactions}. At the outset, ZCash can be considered to be a \textbf{\textit{fork of Bitcoin}.} 
\end{itemize}

\subsection{Toy ZCash}
For the sake of simplicity, consider \textit{Toy ZCash}. The difference between toy ZCash and real ZCash is that it assumes that all the coins have a \textbf{fixed nomination}. Toy ZCash uses crypto-primitives like zk-SNARK and commitment. 

\underline{\textbf{Commitment}}
Let's say we have a random sample $r$. 
$$r \xleftarrow{\$} D$$
$$ C := \text{COMM}_r(m) $$
Where m is a message. \\
Lets say Bob sends a commitment to Alice. The point of a commitment is that at any point in time, Bob has a secret $r$. If Bob wants to prove that he \textit{committed}, he just needs to review $r$ and $m$ and verify it.  \\

\textbf{How is a new coin created by Toy ZCash?} \\ 
Here, we assume that there is an underlying bitcoin like ledger. Coin minting essentially means that the base coin is converted into private coin. Consider a user $u$: 
$$ r \xleftarrow{\$}{} D$$ 
The user also needs to sample a random number which would be a serial number $sn$ of the coin. 
$$ sn \xleftarrow{\$} D$$, After which a coin commitment needs to be computed. 
$$ cm := \text{COMM}_r(sn)$$
Here, the coin is a triple as follows:
$$ c := (r, sn, cm) $$
Where $cm$ is the commitment and $r$ and $sn$ act like a secret. User $u$ sends $TX_{mint}$ (which contains $cm$) as well as one basecoin to the ledger. When the ledger receives this $cm$ and one basecoin, $TX_{mint}$ is appended to the ledger. Note that, the user need to disclose their identity/address in this case. The transaction can take place over the underlying peer-to-peer network and thus a simple message relay would suffice. \\

The ledger contains $CM_{list}$ which is a list of all coin commitments. To increase efficiency, a merkle tree can be used instead of a simple list. \\

\textbf{What is the same secret is used by another user $u'$?} \\
Lets say $u'$ has the same secret $c := (r, sn, cm)$ and wants to \textit{spend} the coin. This is equivalent to \textit{claiming a base coin from private coin.} $u'$ could send $c$ by posting $TX_{spend}$. $TX_{spend}$ has the following parts to it: 
\begin{itemize}
    \item Serial number $sn$.
    \item zk-Proof of NP Statement $\pi_i$. $u'$ needs to prove the NP Statement having 2 parts:
    \begin{itemize}
        \item An $r$ is known, such that $cm' = \text{COMM}_r(sn)$
        \item $cm' \in CM_{list}$
    \end{itemize}
    Here, both $r$ and $cm'$ are private. 
\end{itemize}

Both the sender and the receiver shouldn't reveal the secret $r$. If the secret $r$ is revealed, everyone can track the transactions. The \textit{\underline{key}} is here that $r$ \textit{\textbf{should always be secret}} and can only be shared between the sender and receiver. 
\subsection{Optimizations}
\textbf{Optimization 1:} \\
As part of the NP Statement, $cm' \in CM_{list}$ needs to be proved. The size of the $CM_{list}$ grows linearly. This lowers the look-up performance as the list grows. The use of a \textit{\textbf{Merkle Tree}} helps solve this problem. So now, we only need to prove that $cm' \in R_t$ where $R_t$ is the merkle root. This reduces the look-up time $O(n)$ to $O(log n)$.\\

\textbf{Optimization 2:} \\
\textbf{Direct payment}: Lets say we have 2 new requirements: 
\begin{itemize}
    \item Support arbitrary value (unlike toy ZCash).
    \item Sender shouldn't be able to track the coin after sending, i.e., the sender shouldn't know when the receiver spends the coin using the serial number $sn$ that the sender sampled.
\end{itemize}
We can \textbf{redesign} the \textbf{\textit{commitment scheme}} in order to meet these requirements as follows: \\
Derive 3 PRFs from a single random seed: $x$
\begin{itemize}
    \item $PRF_x^{addr}(\cdot)$ is used to compute the address
    \item $PRF_x^{sn}(\cdot)$ is used to compute the serial number
    \item $PRF_x^{PK}(\cdot)$ is used to compute the public key, ensuring that this function is collision resistant. 
\end{itemize}

\textbf{\textit{Address Scheme}}: User $u$ generates a key-pair $(a_{pk}, a_{sk})$ with a constraint the a coin designated to $a_{pk}$ can only be spent by someone having $a_{sk}$. This coin could be user $u$'s own coin which they got by minting or a coin sent by some other user using the public key $a_pk$. The key-pair is generated as follows: 
$$a_{sk} \xleftarrow{\$} A$$
$$a_{pk} := PRF_{a_{sk}}^{addr}(\cdot)$$ \\
\textbf{Redesign of Minting scheme} \\
Goal here is to mint a coin $c$ with value $v$. User has a secret $a_{sk}$. Let $\rho$ be the coin specific random secret.
$$\rho \xleftarrow{\$} D$$
$$sn := PRF_{a_{sk}}^{sn}(\rho)$$
This coin is signed by both the coin specific secret($\rho$) and user-specific secret $a_{sk}$. Now this is a 2-stage commitment. 
\begin{itemize}
    \item $k:=\text{COMM}_r(a_{pk} || \rho)$, $r \xleftarrow{\$}D$
    \item $cm:=\text{COMM}_s(v||k)$, $s \xleftarrow{\$}D$, commit the value as well, since at a certain point in time, this value has to be revealed. 
\end{itemize}
To formally define the coin $c$, \\
$$c:=(a_{pk}, v, \rho, r, s, cm)$$
The coin has public key, value, random serial number, secret of first stage commitment, secret of second stage commitment and the final commitment.  
$$TX_{mint} := (v, k, s, cm)$$

\textbf{Observations}
\begin{itemize}
    \item In the final mint transaction, it has cm, s and v. Everyone can verify whether $cm =? \text{ COMM}_s(v||k)$ 
    \item $\rho, a_{pk}$ are hidden 
\end{itemize}

\textbf{\underline{Note:}} Mint transaction is only valid if user $u$ deposits $v$ BTC.

\subsection{Pour Protocol}
The spending protocol is called the pour protocol in this case. In order to upkeep the privacy guarantee of the transaction, linking any 2 coins just by observing their values should be avoided. This is done by maintaining the following relation:
$$v_{old} = v_1 + v_2$$
Here, $v_{old}$ is the \textit{coin-in} and $v_1$ and $v_2$ are the coins out. Effectively, we now have 3 coins such that:
$$c^{old} = (a_{pk}^{old}, v^{old}, \rho^{old}, r^{old}, s^{old}, cm^{old}) $$
$$c_1^{new} = (a_{pk,1}^{new}, v_1^{new}, \rho_1^{new}, r_1^{new}, s_1^{new}, cm_1^{old})$$
$$c_2^{new} = (a_{pk,2}^{new}, v_2^{new}, \rho_2^{new}, r_2^{new}, s_2^{new}, cm_2^{old})$$

User wishes to pour $c^{old}$ to $c_1^{new}, c_2^{new}$. Following the 2 stage commitment rules, 
$\rho$ is generated as follows:
$$\rho_i^{new} \xleftarrow{\$}D, \text{ where } i \in {1, 2}$$
\begin{itemize}
    \item $r_i^{new} \xleftarrow{\$}D$, $k_i^{new}=\text{COMM}_{r_{i}}^{new}(a_{pk, i}^{new} || \rho_i^{new})$
    \item $s_i^{new} \xleftarrow{\$} D$, $cm_i^{new} := \text{COMM}_{s_{i}}^{new}(v_i^{new} || k_i^{new})$
\end{itemize}
The serial number of the old coin ($sn_{old}$) while pouring, is \textbf{public}. 
Essentially, serial number is used to prevent double spending; recall that the ledger is checked if $sn$ was spent previously. $cm_1^{new}$ and $cm_2^{new}$ is public since they are written to the ledger. Everything else is kept private. 

\textbf{zk-Statement:} I know $c_{old}, c_1^{new}, c_2^{new}, \rho^{old} \text{ and } a_{sk}^{old}$ such that: 
\begin{itemize}
    \item  $cm^{old} \in \text{TREE}(root)$
    \item  $v_{old} = v_1^{new} + v_2^{new}$
    \item  $sn^{old} = PRF_{a_{sk^{old}}}^{sn}(\rho)$, this step verifies the serial number $sn^{old}$
    \item $cm_i^{new}$ is well formed, where $i \in {1, 2}$
    \item $a_{pk}^{old} = PRF_{a_{sk}^{old}}^{addr}(\cdot)$  Intuition: Before spending the coin, the user needs to prove that they are the owners of the old coin. 
\end{itemize}

\subsection{Summary}
To summarize, ZCash is the first payment system which truly provides privacy to the transactions in a distributed system. By the scheme of the ZCash protocol, the sender has no way to track the transaction and the coin through the ZCash system. 

\end{document}


