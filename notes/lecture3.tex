%
% This is a borrowed LaTeX template file for lecture notes for CS267,
% Applications of Parallel Computing, UCBerkeley EECS Department.
% Now being used for CMU's 10725 Fall 2012 Optimization course
% taught by Geoff Gordon and Ryan Tibshirani.  When preparing 
% LaTeX notes for this class, please use this template.
%
% To familiarize yourself with this template, the body contains
% some examples of its use.  Look them over.  Then you can
% run LaTeX on this file.  After you have LaTeXed this file then
% you can look over the result either by printing it out with
% dvips or using xdvi. "pdflatex template.tex" should also work.
%

\documentclass{article}
\setlength{\oddsidemargin}{0.25 in}
\setlength{\evensidemargin}{-0.25 in}
\setlength{\topmargin}{-0.6 in}
\setlength{\textwidth}{6.5 in}
\setlength{\textheight}{8.5 in}
\setlength{\headsep}{0.75 in}
\setlength{\parindent}{0 in}
\setlength{\parskip}{0.1 in}

%
% ADD PACKAGES here:
%

\usepackage{amsmath,amsfonts,graphicx}

%
% The following commands set up the lecnum (lecture number)
% counter and make various numbering schemes work relative
% to the lecture number.
%
\newcounter{lecnum}
\renewcommand{\thepage}{\thelecnum-\arabic{page}}
\renewcommand{\thesection}{\thelecnum.\arabic{section}}
\renewcommand{\theequation}{\thelecnum.\arabic{equation}}
\renewcommand{\thefigure}{\thelecnum.\arabic{figure}}
\renewcommand{\thetable}{\thelecnum.\arabic{table}}

%
% The following macro is used to generate the header.
%
\newcommand{\lecture}[4]{
   \pagestyle{myheadings}
   \thispagestyle{plain}
   \newpage
   \setcounter{lecnum}{#1}
   \setcounter{page}{1}
   \noindent
   \begin{center}
   \framebox{
      \vbox{\vspace{2mm}
    \hbox to 6.28in { {\bf UCSB CS 291D: Blockchains and Cryptocurrencies
	\hfill Fall 2020} }
       \vspace{4mm}
       \hbox to 6.28in { {\Large \hfill Lecture #1: #2  \hfill} }
       \vspace{2mm}
       \hbox to 6.28in { {\it Lecturer: #3 \hfill Scribes: #4} }
      \vspace{2mm}}
   }
   \end{center}
   \markboth{Lecture #1: #2}{Lecture #1: #2}

%   {\bf Note}: {\it LaTeX template courtesy of UC Berkeley EECS dept.}

%   {\bf Disclaimer}: {\it These notes have not been subjected to the
%   usual scrutiny reserved for formal publications.  They may be distributed
%   outside this class only with the permission of the Instructor.}
%   \vspace*{4mm}
}
%
% Convention for citations is authors' initials followed by the year.
% For example, to cite a paper by Leighton and Maggs you would type
% \cite{LM89}, and to cite a paper by Strassen you would type \cite{S69}.
% (To avoid bibliography problems, for now we redefine the \cite command.)
% Also commands that create a suitable format for the reference list.
\renewcommand{\cite}[1]{[#1]}
\def\beginrefs{\begin{list}%
        {[\arabic{equation}]}{\usecounter{equation}
         \setlength{\leftmargin}{2.0truecm}\setlength{\labelsep}{0.4truecm}%
         \setlength{\labelwidth}{1.6truecm}}}
\def\endrefs{\end{list}}
\def\bibentry#1{\item[\hbox{[#1]}]}

%Use this command for a figure; it puts a figure in wherever you want it.
%usage: \fig{NUMBER}{SPACE-IN-INCHES}{CAPTION}
\newcommand{\fig}[3]{
			\vspace{#2}
			\begin{center}
			Figure \thelecnum.#1:~#3
			\end{center}
	}
% Use these for theorems, lemmas, proofs, etc.
\newtheorem{theorem}{Theorem}[lecnum]
\newtheorem{lemma}[theorem]{Lemma}
\newtheorem{proposition}[theorem]{Proposition}
\newtheorem{claim}[theorem]{Claim}
\newtheorem{corollary}[theorem]{Corollary}
\newtheorem{definition}[theorem]{Definition}
\newenvironment{proof}{{\bf Proof:}}{\hfill\rule{2mm}{2mm}}

% **** IF YOU WANT TO DEFINE ADDITIONAL MACROS FOR YOURSELF, PUT THEM HERE:

\newcommand\E{\mathbb{E}}

\begin{document}
%FILL IN THE RIGHT INFO.
%\lecture{**LECTURE-TITLE**}{**DATE**}{**LECTURER**}{**SCRIBE**}
\lecture{3}{Consensus I: Byzantine General Problems, Dolev-Strong}{Shumo Chu}{Kerem Celik, Radha Kumaran}
%\footnotetext{These notes are partially based on those of Nigel Mansell.}

\section{Byzantine Generals Problem}
The Byzantine Generals Problem is a thought experiment used to model how distributed systems can reach agreement even in adverse conditions. 

\subsection{Allegory}
There exists a group of generals devising a military strategy. Suppose one of the generals is a \textit{commanding general} that proposes the order. The generals, via passing messages, must agree to collectively attack or retreat, in the presence of disloyal generals such that two properties hold:
\begin{enumerate}
    \item If the commanding general is loyal, all generals agree with their order
    \item All loyal generals agree on order
\end{enumerate}


\subsection{Formal Definition}
\textbf{Distributed consensus:} Problem in which multiple processes must agree on some value in the presence of failures

\textbf{Byzantine fault}: Condition of a component where it can fail in an arbitrary way

\textbf{Synchronous network:} There is a latency upper-bound for message delivery. If an honest node sends $msg$ in round $r$, the recipient will receive $msg$ in round $r+1$.

\textbf{Authenticated setting:} There exists a public key infrastructure (PKI) where messages can be signed by a node and verified by others with its public key.

The Byzantine Generals Problem, also known as Byzantine Broadcast, requires a solution to allow distributed consensus in a network where nodes can suffer Byzantine faults. Specifically, in a network of $n$ nodes, with one selected as the designated sender, in the presence of honest nodes, which follow the protocol, and $f < n$ corrupt nodes, which can undergo Byzantine faults and whose existence is not known beforehand, the following properties must hold:
\begin{enumerate}
    \item If the designated sender is loyal, all nodes agree with its value
    \item All honest nodes must agree on same value
\end{enumerate}

\newpage
\subsection{Naive Solution}
Suppose we describe a round-based authenticated Byzantine Broadcast protocol where a designated sender receives bit $b$ as input and all nodes must output a bit after the protocol to signify consensus. We assume message validation using a PKI is done upon every method receipt and sent messages are signed.
\begin{enumerate}
    \item [] Round 1. Designated sender receives bit $b$ as input and broadcasts it
    \item [] Round 2. For each node, if it receives a single bit $b'$, broadcasts the vote $b'$ else broadcasts the vote $0$
    \item [] Round 3. Output the bit that got majority vote else output $0$
\end{enumerate}

A simple attack can be constructed for this protocol. Suppose there are three nodes, $S, C_0, C_1$ where $S$ is the designated sender and is corrupt. 
\begin{enumerate}
    \item [] Round 1. $S$ sends $0$ to $C_0$ and $1$ to $C_1$
    \item [] Round 2. $C_0$ votes $0$, $C_1$ votes $1$. $S$ votes $0$ to $C_0$ and $1$ to $C_1$
    \item [] Round 3. $C_0$ counts $2$ votes for $0$ and outputs 0. $C_1$ counts $2$ votes for $1$ and outputs 1.
\end{enumerate}

\section{The Dolev-Strong Protocol}

The Dolev-Strong Protocol is a round-based protocol that solves the Byzantine Generals problem in a synchronous and authenticated setting.

\subsection{Setup}
There are $n$ nodes numbered $1, 2,\dots, n$, and we assume that node $1$ is the designated sender. Each node $i$ maintains an extracted set $extr_i$ which contains all the distinct valid bits chosen so far. $\langle b \rangle_S$ denotes a bit $b$ that has valid signatures by the set of nodes $S, S \subseteq [n]$. $f$ denotes an upper bound on the number of corrupt nodes.

\subsection{Protocol}


\begin{itemize}
	\item \textbf{Round $0$}: Sender sends $\langle 1 \rangle_1$ to all nodes.

	\item \textbf{For each round $r = 1$ to $f+1$}:

	For every message $\langle \tilde{b} \rangle_{1, j_1, j_2, \dots, j_{r-1}}$ that node $i$ receives with $r$ signatures from distinct nodes:

	\hspace{0.5in} If $\tilde{b} \notin extr_i$:

	\hspace{1in}Add $\tilde{b}$ to $extr_i$.

	\hspace{1in}Send $\langle \tilde{b} \rangle_{1, j_1, j_2, \dots, j_{r-1}, i}$ to everyone.

	\item \textbf{At the end of round $f+1$}:

	If $|extr_i| = 1$, node $i$ outputs the bit in $extr_i$, else node $i$ outputs 0.
\end{itemize}


\subsection{Intuition}
If the nodes had to output at the end of $f$ rounds rather than $f+1$, then the following attack could be constructed, where the sender is one of the $f$ corrupt nodes:

\begin{itemize}
	\item \textbf{Round $0$}: Sender sends $\langle 1 \rangle_1$ to all nodes.

	\item \textbf{For each round $r = 1$ to $f-1$}:	The corrupt nodes send no messages.

	\item \textbf{Round $f$}:

	The set of corrupt nodes $\mathcal{F}$ choose an honest node $v$ and make $v$ receive $\langle 0 \rangle_{i_0, i_1, \dots, i_f}$ ($i_0, \dots, i_f \in \mathcal{F}$).

	\item \textbf{At the end of round $f$}:

	$extr_v = {0, 1}$, so $v$ outputs 0.
	$extr_j = {1}$ for all honest nodes $j$, so they output 1.
\end{itemize}

However, when the algorithm runs for one more round, each message must have $f+1$ signatures. This means at least one honest node (say node $i$) must have signed the message in round $r < f+1$, and propagated this message with $r+1$ signatures to all the other nodes. Therefore, all the honest nodes would have added $b$ to their extracted sets at the beginning of round $r+1$.

% **** THIS ENDS THE EXAMPLES. DON'T DELETE THE FOLLOWING LINE:

\end{document}
