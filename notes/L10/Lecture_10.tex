%
% This is a borrowed LaTeX template file for lecture notes for CS267,
% Applications of Parallel Computing, UCBerkeley EECS Department.
% Now being used for CMU's 10725 Fall 2012 Optimization course
% taught by Geoff Gordon and Ryan Tibshirani.  When preparing 
% LaTeX notes for this class, please use this template.
%
% To familiarize yourself with this template, the body contains
% some examples of its use.  Look them over.  Then you can
% run LaTeX on this file.  After you have LaTeXed this file then
% you can look over the result either by printing it out with
% dvips or using xdvi. "pdflatex template.tex" should also work.
%

\documentclass{article}
\setlength{\oddsidemargin}{0 in}
\setlength{\evensidemargin}{0 in}
\setlength{\topmargin}{-0.6 in}
\setlength{\textwidth}{6.5 in}
\setlength{\textheight}{8.5 in}
\setlength{\headsep}{0.75 in}
\setlength{\parindent}{0 in}
\setlength{\parskip}{0.1 in}

%
% ADD PACKAGES here:
%

\usepackage{amsmath,amsfonts,graphicx}

%
% The following commands set up the lecnum (lecture number)
% counter and make various numbering schemes work relative
% to the lecture number.
%
\newcounter{lecnum}
\renewcommand{\thepage}{\thelecnum-\arabic{page}}
\renewcommand{\thesection}{\thelecnum.\arabic{section}}
\renewcommand{\theequation}{\thelecnum.\arabic{equation}}
\renewcommand{\thefigure}{\thelecnum.\arabic{figure}}
\renewcommand{\thetable}{\thelecnum.\arabic{table}}

%
% The following macro is used to generate the header.
%
\newcommand{\lecture}[4]{
   \pagestyle{myheadings}
   \thispagestyle{plain}
   \newpage
   \setcounter{lecnum}{#1}
   \setcounter{page}{1}
   \noindent
   \begin{center}
   \framebox{
      \vbox{\vspace{2mm}
    \hbox to 6.28in { {\bf UCSB CS 291D: Blockchains and Cryptocurrencies
	\hfill Fall 2020} }
       \vspace{4mm}
       \hbox to 6.28in { {\Large \hfill Lecture #1: #2  \hfill} }
       \vspace{2mm}
       \hbox to 6.28in { {\it Lecturer: #3 \hfill Scribes: #4} }
      \vspace{2mm}}
   }
   \end{center}
   \markboth{Lecture #1: #2}{Lecture #1: #2}

%   {\bf Note}: {\it LaTeX template courtesy of UC Berkeley EECS dept.}

%   {\bf Disclaimer}: {\it These notes have not been subjected to the
%   usual scrutiny reserved for formal publications.  They may be distributed
%   outside this class only with the permission of the Instructor.}
%   \vspace*{4mm}
}
%
% Convention for citations is authors' initials followed by the year.
% For example, to cite a paper by Leighton and Maggs you would type
% \cite{LM89}, and to cite a paper by Strassen you would type \cite{S69}.
% (To avoid bibliography problems, for now we redefine the \cite command.)
% Also commands that create a suitable format for the reference list.
\renewcommand{\cite}[1]{[#1]}
\def\beginrefs{\begin{list}%
        {[\arabic{equation}]}{\usecounter{equation}
         \setlength{\leftmargin}{2.0truecm}\setlength{\labelsep}{0.4truecm}%
         \setlength{\labelwidth}{1.6truecm}}}
\def\endrefs{\end{list}}
\def\bibentry#1{\item[\hbox{[#1]}]}

%Use this command for a figure; it puts a figure in wherever you want it.
%usage: \fig{NUMBER}{SPACE-IN-INCHES}{CAPTION}
\newcommand{\fig}[3]{
			\vspace{#2}
			\begin{center}
			Figure \thelecnum.#1:~#3
			\end{center}
	}
% Use these for theorems, lemmas, proofs, etc.
\newtheorem{theorem}{Theorem}[lecnum]
\newtheorem{lemma}[theorem]{Lemma}
\newtheorem{proposition}[theorem]{Proposition}
\newtheorem{claim}[theorem]{Claim}
\newtheorem{corollary}[theorem]{Corollary}
\newtheorem{definition}[theorem]{Definition}
\newenvironment{proof}{{\bf Proof:}}{\hfill\rule{2mm}{2mm}}

% **** IF YOU WANT TO DEFINE ADDITIONAL MACROS FOR YOURSELF, PUT THEM HERE:

\newcommand\E{\mathbb{E}}

\begin{document}
%FILL IN THE RIGHT INFO.
%\lecture{**LECTURE-TITLE**}{**DATE**}{**LECTURER**}{**SCRIBE**}
\lecture{10}{zk-SNARK Primer I}{Shumo Chu}{Gwyneth Allwright, Lianke Qin}
%\footnotetext{These notes are partially based on those of Nigel Mansell.}

% **** YOUR NOTES GO HERE:

% Some general latex examples and examples making use of the
% macros follow.  
%**** IN GENERAL, BE BRIEF. LONG SCRIBE NOTES, NO MATTER HOW WELL WRITTEN,
%**** ARE NEVER READ BY ANYBODY.

\section{Recap of Previous Discussions}
\begin{itemize}
\item Layer 1: 
	\begin{itemize}
	\item Consensus.
	\item Smart contracts.
	\end{itemize}
\item Layer 2:
	\begin{itemize}
	\item Oracles.
	\end{itemize}
\item Universal verifiability: all data is public.
\item Assets are controlled by signatures. Secret keys can be thought of as specialized zero-knowledge proof systems.
\end{itemize}

\section{Introduction to zk-SNARKs}
\begin{itemize}
\item The main benefit of zk-SNARKs is privacy. zk-SNARKs allow us to encrypt public data on blockchains.
\item Toy example: we move from state \texttt{i} to state \texttt{i+1} on the blockchain using the transaction $T$. 
	\begin{itemize}
	\item We encrypt \texttt{i}, \texttt{i+1} and $T$.
	\item We wish to verify that $T$ is a sound transaction for state \texttt{i} and that it produces state \texttt{i+1}. 
	\item This verification should take the form of a zero-knowledge proof $\Pi$.
	\end{itemize}
\end{itemize}

\section{Introduction to Circuits}
\begin{itemize}
\item Boolean circuits are composed using the binary operators AND, OR and NOT.
	\begin{itemize}
	\item AND$(x, y) = x\cdot y$
	\item OR$(x, y) = x+y-x\cdot y$
	\item NOT$(x) = 1-x$ 
	\item We can represent any boolean function as a circuit $C: \{0,1\}^N \longrightarrow \{0,1\}$.
	\end{itemize}
\item Arithmetic circuits are mappings $C: F_p^N \longrightarrow F_p$, where $F_p$ is a finite field.
	\begin{itemize}
	\item Intuitively, finite fields are finite sets of objects with certain operations defined on them.
	\item For example, there is an addition operation. A property of finite fields is that
		\begin{equation}
		a, b \in F_p \implies a+b \in F_p.
		\end{equation}
	\item Arithmetic circuits can be represented using directed acyclic graphs (DAGs).
	\item You can express boolean circuits as arithmetic circuits (but it may be inefficient to do so).
	\end{itemize}
\item The link between circuits and hash functions is as follows. Imagine that we have a circuit
	\begin{equation}
	C_{\text{hash}}: (h, m) \longrightarrow \{0,1\},
	\end{equation}
	which is zero if $h=H(m)$ and unity otherwise, where $H$ is a hash function (e.g. SHA256).
\end{itemize}

\section{Milestones in the History of Proof Systems}
\begin{itemize}
\item In 1992, it was proven that the complexity class IP is equivalent to the class PSPACE (Fortnow, Karloff, Nisan, Shamir).
\item The PCP theorem was proven (Arora, Lund, Safra, Sudan). Idea: any NP statement (anything expressible in a circuit) has a probabilistically checkable proof such that this proof can be verified in polynomial time relative to the size of the classic proof.
\item Developments in non-interactive proof systems (Kilian, Micali, Groth).
\end{itemize}

\section{Illustration of a Scheme for Zero-Knowledge Proofs}
\begin{itemize}
\item In this scenario, we have a prover (Alice) and a verifier (Bob).
\item Imagine that there are two finite field elements $x\in F_p^n$ and $w \in F_p^m$.
	\begin{itemize}
	\item $x$ is public, while $w$ (the witness) is private (known only to Alice).
	\end{itemize}
\item Alice uses $x$ and $w$ to compute the arithmetic circuit $C(x, w) \in F_p$.
	\begin{itemize}
	\item The circuit $C$ is public.
	\item Note that $C$ could be a representation of a boolean circuit. We will assume that it evaluates to \texttt{true}.
	\end{itemize}
\item Alice wishes to prove the following to Bob:
	\begin{itemize}
	\item Soundness: there exists a $w$ such that $C(x, w)$ evaluates to \texttt{true}.
	\item Knowledge: Alice knows the $w$ in question.
	\end{itemize}
\item Why can't Alice just send $w$ to Bob? Three reasons:
	\begin{itemize}
	\item $w$ needs to be private.
	\item $w$ may be too long to send.
	\item $C(x, w)$ may be inefficient to compute.
	\end{itemize}
\item Instead of sending $w$, Alice will send a zero-knowledge proof $\Pi$. This will be constructed as follows:
	\begin{itemize}
	\item The key elements of the construction are mappings $S$, $P$ and $V$. $S$ is used to generate keys, while $P$ and $V$ are used for proving and verifying respectively.
	\item First, we act $S$ on the circuit $C$ to obtain a pair of keys, $p_k$ (proving key) and $v_k$ (verification key).
	\item Alice computes $P(p_k, x, w)$ to obtain a proof $\Pi$.
	\item Bob computes $V(v_k, x, \Pi)$ to obtain either \texttt{true} (if the proof was correct) or \texttt{false}.
	\item In order for this scheme to be successful, we require that
		\begin{equation}
		V(v_k, x, \Pi) = \texttt{true} \implies V(v_k, x, P(p_k, x, w)) = \texttt{true},
		\end{equation}
		as well as the fact that $V(v_k, x, \Pi) = \texttt{true} \implies$ Alice knows $w$ such that $C(x, w)$ is \texttt{true}.
	\item This proof is zero-knowledge: $\Pi$ and $x$ reveal nothing about $w$.
	\end{itemize}
\end{itemize}


%Lianke starts here

% \section{Finite Field}
% \begin{itemize}
%     \item A finite size of set whose size is a prime number.
%     \item Finite field is addition complete : for any $a \in F$ and $b \in F$ \longrightarrow $a + b \in F$.
%     \item for multiplication in finite field : 
% \end{itemize}


\section{Proof of Knowledge}
In cryptography, a proof of knowledge is an interactive proof in which the prover succeeds in ``convincing'' a verifier that the prover knows something. What it means for a machine to ``know something'' is defined in terms of computation. Knowledge extractor is introduced to capture this idea.
\begin{itemize}
    \item $\pi$ : its size could be 2KB.
    \item $P$ is used for proving
    \item $V$ is used for verifying
    \item $S$ is used to generate keys.
    \item Extractor


\end{itemize}


Then we say ($S$, $P$, $V$) is a proof of knowledge that satisfies:


\begin{itemize}

        \item ${\forall}$ c, s.t.  ${\forall}$ unbounded adversary. $A = (A_0, A_1) $
        \begin{itemize}
            \item  $(pk, vk) := S(C)$
            \item  $(x, st) := A_0(pk)$ is to generate a fake $w$
            \item  $\pi^{\prime} := A_1(pk, x, st)$ is to generate a fake $\pi$
        \end{itemize}

        \item s.t.  $Pr[V(vk, x, \pi^{\prime}] = true] > negl(\cdot) $
        \item There is an efficient Extractor(use $A$ as black box),
        \begin{itemize}
            \item $(pk, vk) := S(C)$, $(x, st) := A_0(pk)$
            \item $w := E(pk, x, st)$
            \item $Pr[ C(x, w) = 0 ] > negl(\cdot)$
        \end{itemize}
        \item Argument of knowledge : $A$ is $poly(\cdot)$

\end{itemize}



\section{Zero Knowledge}

By zero knowledge we want to ensure :
\begin{itemize}
    \item ($x$, $\pi$) did not reveal $w$.
    \item ($S$, $P$, $V$) is a zero knowledge proof for circuit $C$.
    \item There is an efficient simulator s.t.:
    \begin{itemize}
        \item $\forall$ $x \in F_p^{n}$, $\exists$ $w$ : $C(x, w) = 0$, we have :
        \item ($pk$, $vk$, $x$, $\pi$) where $(pk, vk) := S(C)$ and $\pi := P(pk, x, w)$ is indistinguishable with :
        \item ($pk$, $vk$, $x$, $\pi$) where $(pk, vk, \pi) := Sim(x)$
        \item This means that $Sim(x)$ can simulate $\pi$ without $w$
    \end{itemize}
\end{itemize}



\section{zk-SNARK protocols}
The acronym zk-SNARK stands for “Zero-Knowledge Succinct Non-Interactive Argument of Knowledge,” and refers to a proof construction where one can prove possession of certain information without revealing that information, and without any interaction between the prover and verifier.

\begin{itemize}
    \item Zero-Knowledge has the meaning explained above: nothing is revealed beyond truth of statement to the verifier.
    \item Succinct means that the proof is tiny compared to the computation.
    \begin{itemize}
        \item The proof size is constant $O(1)$.
        \item Verification time is $O(1)$ and does not depend on the running time of $f$.
    \end{itemize}
    \item Non-interactive means that we can write and store a proof, without the need to have question/answer cycles. So a proof can be computed and published and everyone can verify it.
    \item ARgument of Knowledge means that soundness is guaranteed only against a computationally bounded server and the proof cannot be constructed without access to a witness.
\end{itemize}


There is a simple comparison among several popular zk-SNARK protocols below:
\begin{table}[ht]
\centering
\begin{tabular}{|c|c|c|c|c|}
        \hline
        & Size of $\pi$ & Size of $pk$ & verification time & setup           \\ 
        \hline
Groth16~\cite{1} & O(1)                       & O(log$|C|$)  & O(1)              & Yes/Per circuit \\
        \hline
PLONK~\cite{2}   & O(1)                       & O(log$|C|$)  & O(1)              & Yes/Update      \\
        \hline
STARK~\cite{3}   & O(log$|C|$)                  & O(1)       & O(log$|C|$)         & No       \\      
        \hline

\end{tabular}
\caption{Mostly recently used zk-SNARK protocols} 

\end{table}

\section*{References}
\beginrefs
\bibentry{1} 
Groth, Jens. "On the size of pairing-based non-interactive arguments." Annual international conference on the theory and applications of cryptographic techniques. Springer, Berlin, Heidelberg, 2016.
\bibentry{2}
Gabizon, Ariel, Zachary J. Williamson, and Oana Ciobotaru. "PLONK: Permutations over Lagrange-bases for Oecumenical Noninteractive arguments of Knowledge." IACR Cryptol. ePrint Arch. 2019 (2019): 953.
\bibentry{3}
Ben-Sasson, Eli, et al. "Scalable, transparent, and post-quantum secure computational integrity." IACR Cryptol. ePrint Arch. 2018 (2018): 46.
\endrefs

% **** THIS ENDS THE EXAMPLES. DON'T DELETE THE FOLLOWING LINE:

\end{document}

